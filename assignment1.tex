\documentclass[10pt,a4paper]{article}
\usepackage[utf8]{inputenc}
\usepackage{amsmath}
\usepackage{amsfonts}
\usepackage{multicol}
\usepackage{amssymb}
\usepackage[framed]{matlab-prettifier}
\usepackage{graphicx}
\usepackage[margin=0.75in]{geometry}
\usepackage{enumerate}
\usepackage{circuitikz}
\usepackage[utf8]{inputenc}
\usepackage[T1]{fontenc}
\usepackage{karnaugh-map}
\usepackage{tikz}
\usetikzlibrary{shapes.gates.logic.US}
\usetikzlibrary{circuits.ee.IEC}

\begin{document}
\begin{center}

{\huge EE5811 : FPGA LAB}\\
{\large ASSIGNMENT 1}

\end{center}
\begin{center}
    
Yeleboina Venkata Lavanya 
\\EE21MTECH11015
\end{center}

\vspace{15pt}
\hrule
\vspace{5pt}
\section*{Question}
Convert the following boolean expression into its canonical POS form
\begin{equation*}
    F(A,B,C)=(B+\Bar{C}).(\Bar{A}+B)
\end{equation*}




\section{Solution}
\subsection{Using boolean laws}
The given function F has three variables A,B and C.Each OR term in the given expression has one missing variable
\\From boolean laws,

\begin{equation}
x\bar{x}=0
\end{equation}
\\From (1) The given boolean expression can be written as
\begin{equation}
    F(A,B,C)=(B+\Bar{C}+A\bar{A}).(\Bar{A}+B+C\bar{C})
\end{equation}
\\We know, from distributive law
\begin{equation}
(x+yz)=(x+y)(x+z)
\end{equation}
\\Using (3),the expression (2) can be written as
\begin{equation}
    F(A,B,C)=(B+\Bar{C}+A).(B+\Bar{C}+\bar{A})(\Bar{A}+B+C)(\Bar{A}+B+\bar{C})
\end{equation}
\\Removing redundant terms,we obtain canonical POS form as
\begin{equation}
    F(A,B,C)=(A+B+\Bar{C}).(\Bar{A}+B+C)(\Bar{A}+B+\bar{C})
\end{equation}
\\Each OR term in (5) has all three variables.Thus, the function is expressed as product of maxterms.Hence it is the canonical POS(product of sums) form of given boolean expression.


\begin{equation}
F=M_{0}M_{4}M_{5}
\end{equation}
\subsection{Using truth table}
The canonical POS form can also be found using truth table.The truth table for given boolean expression is as follows
\begin{center}
    
    
    \begin{tabular}{ | c | c | c | c | }
    \hline
    A & B & C & F(A,B,C) \\ [0.5ex]
     \hline
    0 & 0 & 0 & 1 \\
    0 & 0 & 1 & 0 \\
    0 & 1 & 0 & 1 \\
    0 & 1 & 1 & 1 \\
    1 & 0 & 0 & 0 \\
    1 & 0 & 1 & 0 \\
    1 & 1 & 0 & 1 \\
    1 & 1 & 1 & 1 \\ [1ex]
    \hline
    \end{tabular}
    
\end{center}
From truth table,
\begin{equation}
  F=m_{0}+m_{2}+m_{3}+m_{6}+m_{7} 
  \end{equation}
\begin{equation}
  F=M_{1}M_{4}M_{5}
\end{equation}
From (8),The canonical POS form of given expression is
\begin{equation*}
  F=(A+B+\bar{C})(\bar{A}+B+C)(\bar{A}+B+\bar{C})
\end{equation*}
\section{Verification using KMAP Implementation}
The obtained canonical POS expression can be minimized using KMap as shown in Figure. 
\begin{figure}[h!]
    \centering
    \begin{karnaugh-map}[4][2][1][][]
        
        \minterms{0,2,3,6,7}
        \maxterms{1,4,5}
        \implicant{4}{5}
        \implicant{1}{5}
        \draw[color=black, ultra thin] (0, 2) --
        node [pos=0.7, above right, anchor=south west] {$BC$} % Y label
        node [pos=0.7, below left, anchor=north east] {$A$} % X label
        ++(135:1);
        
    \end{karnaugh-map}
   
    \label{fig:kmap}
\end{figure}

Using implicants in K-Map,the minimized POS expression is
\begin{equation*}
    F(A,B,C)=(B+\Bar{C}).(\Bar{A}+B)
\end{equation*}



\end{document}
© 2022 GitHub, Inc.
Terms
Privacy
Security
Status
Docs
Contact GitHub
Pricing
API
Training
Blog
About
